%---------------------------------------------------------------------
%
%                          Capítulo 1 - Introducción
%
%---------------------------------------------------------------------

\chapter{Introducción}

En esta seccion, se van a explicar las motivación de este trabajo, los objetivos y el desglose del contenido que se puede encontrar en esta memoria.

%-------------------------------------------------------------------
\section{Motivación}
https://www.overleaf.com/project/5d2ae4fe7e5477218544d2ec
\paragraph{}
En una consulta clínica en la que se trabaja con personas que presentan problemas diagnosticados por un psicólogo para la gestión de la ira, es importante recapitular los episodios de la vida cotidiana de esta persona en las que el paciente en cuestión ha tenido que gestionar la ira. En esta narrativa, la terapeuta irá realizando una serie de apuntes para luego intentar ayudar al paciente a que, en situaciones venideras, dicha gestión de la ira mejore o siga siendo estable. Este método, aunque se puede complementar con tests de medición de la ira, presenta un problema fundamental que es que no hay manera de ratificar la veracidad de lo que está diciendo el paciente, por lo que, ya sea de manera consciente o inconsciente (incluso por olvido se le puede olvidar mencionar algunos de los episodios de ira), puede que el psicólogo acabe dando una serie de pautas al paciente partiendo de unas premisas distintas a las reales. A su vez, las pautas que recibe el paciente para la gestión de la ira puede que no sean útiles, y éste tenga que esperar hasta la siguiente consulta para poder recibir nuevas indicaciones. Por otro lado, la categorización de la utilidad de las pautas suministradas por el terapeuta dependen de la información recibida por el paciente en las consultas, que, como se verá en la sección ~\ref{intervencionPsico}, fiarse del paciente en lo que concierne a la gestión de la ira puede conllevar la realización de un mal diagnóstico del paciente.

\paragraph{}
Así, se ha diseñado una solución tecnológica que permite discretizar los episodios de ira del paciente mediante la medición de varias de sus constantes fisiológicas. Esto permitiría a la psicóloga tener una visión más objetiva de dichos episodios. Por otra parte, esta solución tecnológica debería identificar las pautas que más ayudan al paciente para mostrarle aquella que es más adecuada en cada momento.

\paragraph{}
De esta manera, se tendría un registro preciso de los episodios de ira que ha experimentado el paciente cruzados con las pautas que le han podido ser más útiles así como la fecha y hora concreta del inicio y fin de los episodios. Esto reduce significativamente la dependencia en la versión del paciente, pudiendo de esta manera reconducir la terapia hacia los episodios más significativos recogidos en la aplicación, que en el caso de no llevar la pulsera el paciente pudiera no recordar.

\paragraph{}
Por otro lado, un factor crucial de la aplicación es la fiabilidad en la medición de estos episodios de ira. Estos falsos positivos serán descubiertos bien en la propia consulta, bien a través de los comentarios que pueda haber suministrado el paciente a través de la aplicación. Esto es así porque cada vez que se le presente al paciente una pauta, este podrá opcionalmente dar información por escrito o por voz que permita que en la propia consulta se parta no sólo de la información cuantitativa recogida a través de las constantes fisiológicas, sino también de la información cualitativa provista opcionalmente por el paciente.

%-------------------------------------------------------------------

\section{Objetivos}

\paragraph{}
Los dos objetivos principales de este trabajo son que la aplicación que se cree pueda dar información más precisa tanto al terapeuta como al paciente de los episodios de ira acaecidos y, por otro lado, dar información en tiempo real al paciente que le puedan servir para reducir su nivel de ira.

\paragraph{}
Los objetivos de este trabajo derivados de los dos objetivos principales son los siguientes:
\begin{itemize}
    \item Utilización de elementos tecnológicos no intrusivos para que la discretización de los episodios de ira del paciente no supongan un estigma ni sean incómodos en términos fisiológicos. Esto a su vez permitirá ver la evolución histórica de dicho paciente.
    \item Ser capaz de detectar si el paciente ha desconectado o ha dejado de llevar encima los elementos tecnológicos para la medición de la ira.
    \item Intentar incrementar lo máximo posible la pendiente de la curva de aprendizaje para que, entre otras cosas, los elementos tecnológicos en sí acaben siendo motivos del aumento de la frustración e ira.
    \item Adecuación del consumo energético de los dispositivos electrónicos al uso cotidiano. Esto es, no es factible requerir al usuario que cargue el dispositivo móvil o la pulsera inteligente cada 20 minutos por un uso intensivo de energía en la solución tecnológica.
    \item Representación de la información de los episodios de la ira de tal forma que sea comprensible e intuitiva para el profesional que va a revisar los informes de los pacientes.
    \item Capacidad para detectar correctamente no sólo los episodios de ira, sino también su intensidad, duración y momento cronológico.
    \item Representación adecuada a cada paciente de las pautas para la gestión de la ira.
    \item Diseño centrado en el usuario para ajustarse a las necesidades reales tanto del terapeuta como del paciente.
\end{itemize}

\section{Estructura de la memoria}
\paragraph{}
[TODO: esto queda postergado hasta que esté terminada la primera versión de toda la memoria.]