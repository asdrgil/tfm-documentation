%---------------------------------------------------------------------
%
%                          Capítulo 4 - Descripción del problema
%
%---------------------------------------------------------------------

\chapter{Termómetro de la ira}

En este capítulo se va a detallar el proceso que se ha seguido para llegar a implementar el termómetro de la ira. En la sección~\ref{sec:c4:intro} se introduce el problema al que se desea dar solución con el termómetro de la ira. En la sección~\ref{sec:c4:cap} se explica cómo se ha llevado a cabo la captura de requisitos y cuáles fueron los requisitos iniciales de la aplicación. En la sección~\ref{sec:c4:diseño} se explica cómo se desarrolló el diseño de la aplicación con la ayuda de los usuarios finales y la experta en el tratamiento de la ira. Por último, en la sección~\ref{sec:c4:impl}, se detalla cómo se ha implementado y se muestra el resultado final.

%-------------------------------------------------------------------
\section{Introducción}
\label{sec:c4:intro}
\paragraph{}
A la hora de tratar clínicamente a un paciente con problemas de salud mental, existe el problema de que aquello que cuenta el paciente sobre su problema puede no corresponderse con la realidad. Un paciente puede acudir por voluntad propia o de terceros y tanto su versión como la interpretación de los hechos que le han llevado a la consulta pueden diferir severamente entre cada persona. En el caso concreto de la ira, tal y como se comentaba en la sección~\ref{subsubsec:intervencionPsico}, la ira se desarrolla en entornos muy específicos (y diferentes entre cada paciente), supone una distorsión intencionada de la realidad y, por la deseabilidad social del paciente, puede que niegue que tenga problemas de gestión de la ira a la hora de dirigirse al terapeuta.

\paragraph{}
Por todo ello, se propone una solución tecnológica en la que, por un lado el paciente pueda tener en tiempo real información que le sea de utilidad para gestionar dicho episodio de ir y, por otro, dé información adicional a la terapeuta para guiar la consulta para la gestión de la ira. Se propone un sistema tecnológico que mida de forma continuada las constantes fisiológicas del paciente sin que esto interfiera en su día a día, interprete y detecte los episodios de ira y cuando surja un episodio de ira, la aplicación le dará una serie de pautas (previamente cargadas en el sistema por el terapeuta) que le ayuden a gestionar el episodio de ira para volver a una situación de calma. Toda la información del paciente se irá registrando para su posterior análisis por parte del terapeuta. La información recogida resultará de gran ayuda para detectar si el tratamiento está siendo efectivo (es decir, si los episodios de ira disminuyen en intensidad, duración y frecuencia a lo largo del tiempo) y si las pautas asignadas por el terapeuta son o no efectivas.

\paragraph{}
Para medir las constantes fisiológicas se utilizará una pulsera inteligente, que es un formato de dispositivo inteligente ampliamente usado entre segmentos muy diversos de la población, lo que evitará la estigmatización del paciente ya que nadie tiene por qué saber el uso que se está dando a la pulsera.

\paragraph{}
Es importante recalcar que esta solución tecnológica bajo ningún concepto pretende sustituir la terapia con el paciente. Por contra, es una herramienta adicional que intenta mejorar los resultados de la atención clínica al paciente en cuestión.

\paragraph{}
Debido a que no somos expertos en el tratamieto de la ira, en este proceso será fundamental realizar un diseño centrado en el usuario, en el que se contará en todas las fases del proceso con una persona experta que trabaja con la gestión de la ira.

\section{Captura de requisitos}
\label{sec:c4:cap}
\paragraph{}
Una vez identificado el problema, es fundamental establecer contacto con usuarios finales de la aplicación para capturar los requisitos. En este caso, existen dos perfiles de usuarios finales: los pacientes con ira disfuncional y los terapeutas de dichos pacientes. Para este caso, se ha incorporado el conocimiento de una psicóloga que trabaja con este tipo de pacientes que guiará el diseño no solo en lo que concierne a la información que le será útil al terapeuta sino también a qué y cómo debe presentarse la información a los pacientes con ira disfuncional.

\paragraph{}
Para la captura de requisitos, se han realizado dos reuniones con la psicóloga en las que se han recogido unos requisitos que tengan en cuenta tanto las necesidades de los usuario finales como las restricciones tecnológicas. Así, los requisitos funcionales que se han definido tras estas reuniones son:

\begin{enumerate}
    \item Una pulsera inteligente recogerá las constantes fisiológicas a partir de las cuales se intentará averiguar si dicho paciente está sufriendo ira.
    \item El paciente tendrá una aplicación instalada en su móvil, cuyo sistema operativo debe ser Android.
    \item El terapeuta tendrá una aplicación web donde podrá ver el registro de episodios de sus pacientes así como modificar las pautas asociadas a cada uno de ellos.
    \item Cuando se detecte que el paciente pueda estar sufriendo ira, en el móvil del paciente, deberán aparecerle pautas que le ayuden a gestionar dicha ira hasta que se detecte que el episodio ha terminado (el paciente ha vuelto a un estado de calma).
    \item El grado de ira que se le presentará al paciente en un termómetro horizontal en el que el menor valor esté a la izquierda y el valor de mayor ira esté a la derecha.
    \item El paciente podrá descartar pautas y emitir comentarios sobre las mismas para así poder afinar mejor las pautas que les son útiles a cada paciente.
    \item Se registrará el uso y desuso de las pautas para cada paciente, pudiendo el terapeuta modificarlas convenientemente si viese que estas no están siendo útiles.
    \item Todos los episodios de ira del paciente serán registrados para su posterior revisión por parte del terapeuta y del propio paciente.
    \item Las pautas deben de ser definidas por el terapeuta.
    \item La aplicación móvil para el paciente debe ser muy visual y con una curva de aprendizaje poco elevada, lo que implicará limitar el rango de acciones disponibles para el usuario.
    \item Los textos de la aplicación deberán ser en lectura fácil\footnote{Los textos de lectura fácil son aquellos que se realizan con un vocabulario sencillo para que estos puedan ser comprendidos por personas con discapacidad intelectual.} para que estos puedan ser accesibles para todo tipo de usuarios.
\end{enumerate}

\section{Diseño}
\label{sec:c4:diseño}
\paragraph{}
Una vez establecidos los requisitos que debe cumplir la aplicación, se definió la interfaz y las interacciones. El diseño se ha creado en dos iteracciones que se presentan en detalle en las siguientes subsecciones. En la primera iteracción se utilizaron prototipos en papel porque es una solución que requiere de poco esfuerzo temporal y es muy versátil, a la vez que es independiente de las tecnologías que luego se puedan usar. La ventaja de esto es que los cambios en esta etapa prematura del proceso no suponen una inversión elevada de tiempo en modificaciones del diseño.

\paragraph{}
En la segunda iteracción los prototipos que se presentaron fueron de alta fidelidad y se realizaron con las mismas tecnologías con las que luego se implementaría la solución, es decir, HTML y Android.

\subsection{Primer iteracción}
\paragraph{}
En los prototipos en papel, se incluían recortables de papel que al juntarlos entre sí, formaban las distintas pantallas de la web. Se intentó dar distintas alternativas para la misma funcionalidad o interacción aunque como se verá, existen algunos casos con una única opción. Esto se debe a que se creyó que la solución propuesta era estándar o no podía mejorarse significativamente con otras alternativas.

\subsubsection{Diseño de la aplicación móvil de los pacientes}

\paragraph{}
En la figura~\ref{fig:c4:mockup1} se muestra la pantalla que le aparecería al paciente nada más iniciar la aplicación. Es una pantalla de emparejamiento en la que el paciente tendría que introducir el token que le diera el terapeuta para así poder emparejar la aplicación con el paciente.

\paragraph{}
La pantalla de la figura~\ref{fig:c4:mockup2} es una alternativa a la pantalla~\ref{fig:c4:mockup1} pero con cadenas de caracteres de longitud tres. Más tarde se vio que esta pantalla ofrecía información duplicada respecto a la primera, por lo que en las siguientes versiones no se volvió a repetir su contenido.

\begin{figure}[h]
    \centering
    \begin{minipage}{.45\textwidth}
        \centering
        \includegraphics[width=0.8\linewidth, height=7cm]{Imagenes/anxA1-1.png}
        \caption[Mockup para iniciar sesión en la aplicación (versión I)]{Mockup para iniciar sesión en la aplicación (versión I)}
        \label{fig:c4:mockup1}
    \end{minipage}
    \hfill\vline\hfill
    \begin{minipage}{.45\textwidth}
        \centering
        \includegraphics[width=0.8\linewidth, height=7cm]{Imagenes/anxA1-5.png}
        \caption[Mockup para iniciar sesión en la aplicación (versión II)]{Mockup para iniciar sesión en la aplicación (versión II)}
        \label{fig:c4:mockup2}
    \end{minipage}
\end{figure}

\paragraph{}
En la figura~\ref{fig:c4:mockup3}, aparece la pantalla con el nivel actual de ira del paciente. Si el paciente tiene un nivel de ira elevado (se encuentra inmerso en un episodio de ira) en esta pantalla también aparecen la pauta recomendada y la opción para introducir un comentario (por ejemplo, si decide no seguir la pauta recomendada).

\paragraph{}
La pantalla de la figura~\ref{fig:c4:mockup4} es una alternativa de las pantallas que aparecen en las figuras~\ref{fig:c4:mockup3},~\ref{fig:c4:mockup5} y~\ref{fig:c4:mockup6}, en el que la pantalla en la que aparece el nivel de ira y las pautas recomendadas se combina con la pantalla para sincronizar el dispositivo.

\begin{figure}[h]
    \centering
    \begin{minipage}{.4\textwidth}
        \centering
        \includegraphics[width=0.8\linewidth, height=7cm]{Imagenes/anxA1-2.png}
        \caption[Mockup para ver el estado de ira y la pauta recomendada (versión II)]{Mockup para ver el estado de ira y la pauta recomendada (versión II)}
        \label{fig:c4:mockup3}
    \end{minipage}
    \hfill\vline\hfill
    \begin{minipage}{.4\textwidth}
        \centering
        \includegraphics[width=0.8\linewidth, height=7cm]{Imagenes/anxA1-4.png}
        \caption[Mockup para ver el estado de ira y la pauta recomendada (versión I)]{Mockup para ver el estado de ira y la pauta recomendada (versión I)}
        \label{fig:c4:mockup4}
    \end{minipage}%
\end{figure}

\paragraph{}
En la figura~\ref{fig:c4:mockup5} aparece la pantalla para calibrar la pulsera en la que el paciente debe seleccionar si se va a dormir o a despertar. Una vez sincronizada la pulsera, esta pantalla ya no vuelve a aparecer. Esta pantalla es la única con la que podrá interactuar el usuario hasta que haya sincronizado el dispositivo, puesto que cada persona tiene unas constantes vitales distintas y sin esta información no es posible detectar los episodios de ira del paciente.

\paragraph{}
La pantalla de la figura~\ref{fig:c4:mockup6} es la segunda opción que se dio para la pantalla de la figura~\ref{fig:c4:mockup5}, cuya variación reside principalmente en darle al usuario la información de la hora a la que se acostó y se despertó para que así en caso de que haya sido introducido por error, esta información le ayude a dilucidarlo.

\begin{figure}[h]
    \centering
    \begin{minipage}{.4\textwidth}
        \centering
        \includegraphics[width=0.8\linewidth, height=7cm]{Imagenes/anxA1-3.png}
        \caption[Mockup para calibrar el dispositivo (versión I)]{Mockup para calibrar el dispositivo (versión I)}
        \label{fig:c4:mockup5}
    \end{minipage}
    \hfill\vline\hfill
    \begin{minipage}{.4\textwidth}
        \centering
        \includegraphics[width=0.8\linewidth, height=7cm]{Imagenes/anxA1-6.png}
        \caption[Mockup para calibrar el dispositivo (versión I)]{Mockup para calibrar el dispositivo (versión I)}
        \label{fig:c4:mockup6}
    \end{minipage}
\end{figure}

\paragraph{}
Una vez creados los prototipos en papel, se fijó una reunión con la psicóloga en la que se presentaron las distintas alternativas. Las principales conclusiones que se obtuvo de esta reunión fueron las siguientes:
\begin{itemize}
    \item La manera de emparejar la pulsera y el paciente que aparece en la figura~\ref{fig:c4:mockup2} es apropiada.
    \item Las dos alternativas para calibrar el dispositivo que se observan en las figuras~\ref{fig:c4:mock5} y~\ref{fig:c4:mock6} son apropiadas. No ve ninguna razón de peso para elegir una u otra, por lo que finalmente se optó por la alternativa de la figura~\ref{fig:c4:mockup6}, ya que le facilita más información al usuario y le permite rectificar la acción si pulsó accidentalmente alguno de los botones de la pantalla.
    \item El termómetro es una buena opción deberá aparecer en horizontal, estando el menor nivel de ira a la izquierda y el mayor en el otro extremo del termómetro, siguiendo la direccionalidad de la lectura en castellano.
    \item Para indicar el grado de ira, es mejor hacerlo con palabras (como en la figura~\ref{fig:c4:mockup2}) que de manera numérica (como en la figura~\ref{fig:c4:mockup1}). Los estados que discreticen el grado de ira que verá el paciente deberán ser reducidos para que el paciente pueda identificar su ordinalidad con facilidad. Así, la solución se limitará a cinco estados.
    \item Cuando varíe el grado de ira en el termómetro, no solo debe variar cuánto de lleno se encuentra éste, sino también su color. La terapeuta indicó que, tal y como comenta Valdez (\citeyear{valdez1994effects}), existe una relación estrecha entre emociones y colores, por lo que establecer por ejemplo que el rojo es el estado de calma y el verde es el estado de mayor percepción de la ira es un cambio significativo respecto a la asociación opuesta. En estudios experimentales se ha podido comprobar que los colores de mayor longitud de onda como son el rojo o el amarillo provocan mayor activación que otros de menor longitud de onda como es el verde. A su vez, Jacobs y Sues (\citeyear{jacobs1975effects}) han comprobado en un estudio con los colores rojo, amarillo, verde y azul que existe una mayor asociación del estrés con los colores rojo y amarillo respecto al verde y el azul. Así, a la hora de representar los colores asociados a cada nivel de ira, los colores asociados a estos niveles deberían ser (de menor a mayor nivel de ira): verde, verde-amarillo, amarillo, amarillo-rojo, rojo.
    \item Es importante que el paciente pueda cambiar de pauta si esta no le es útil.
    \item El comentario de las pautas debe ser opcional y se debe poder añadir usando la voz.
    \item Se deben incluir elementos sonoros que incrementen su volumen en función del nivel de ira del paciente para que este pueda ser consciente de que ha aumentado su nivel de ira.
    \item En la pantalla de la gestión de la ira se debe permitir al paciente seleccionar el tipo de situación que ha generado la ira.
\end{itemize}

\subsubsection{Diseño de la aplicación web del terapeuta}

\paragraph{}
El elemento de la figura~\ref{fig:c4:mockup7} sirve para mostrar cómo sería el inicio de sesión.

\paragraph{}
El elemento de la figura~\ref{fig:c4:mockup8} muestra una pantalla estándar para registrar a un paciente. El combo para seleccionar el grupo finalmente fue descartado.

\paragraph{}
El elemento de la figura~\ref{fig:c4:mockup9} muestra cómo se crearían grupos y se modificaría su composición de pacientes.

\paragraph{}
En las figuras~\ref{fig:c4:mockup10} y~\ref{fig:c4:mockup11} se muestran dos alternativas para mostrar las pautas: una única tabla para todos los niveles de activación en la que se cuenta con una columna del porcentaje de efectividad general entre los pacientes y una segunda con el texto de la pauta. La primera opción es más visual y cuenta con menos información: no se incluye la columna del porcentaje de éxito de cada pauta y las pautas están divididas en tablas según el nivel de activación asociado.

\begin{figure}[h]
    \centering
    \begin{minipage}{.4\textwidth}
        \centering
        \includegraphics[width=0.8\linewidth, height=7cm]{Imagenes/anxA2.jpg}
        \caption[Mockup para iniciar sesión en la web]{Mockup para iniciar sesión en la web}
        \label{fig:c4:mockup7}
    \end{minipage}
    \hfill\vline\hfill
    \begin{minipage}{.4\textwidth}
        \centering
        \includegraphics[width=0.8\linewidth, height=7cm]{Imagenes/anxA3.jpg}
        \caption[Mockup para registrar al paciente en la web]{Mockup para registrar al paciente en la web}
        \label{fig:c4:mockup8}
    \end{minipage}
\end{figure}

\paragraph{}
En la figura~\ref{fig:c4:mockup12} aparece una pantalla genérica para modificar el paciente en la que a su vez aparece la información asociada a las pautas que le han ido apareciendo en la aplicación móvil.

\paragraph{}
En las figuras~\ref{fig:c4:mockup13},~\ref{fig:c4:mockup14} y~\ref{fig:c4:mockup15} aparecen las tres opciones que se presentaban a la terapeuta para cada paciente en el que se podía ver el resumen de episodios de cada paciente. Dos de ellos consisten en tablas en las que en una aparece las pautas que se recomendaban y en el otro si la pauta suministrada funcionó o no. La opción alternativa era mostrar una gráfica con los distintos episodios acaecidos en un periodo de tiempo.


\begin{figure}[h]
    \centering
    \begin{minipage}{.45\textwidth}
        \centering
        \includegraphics[width=0.8\linewidth, height=7cm]{Imagenes/anxA4.jpg}
        \caption[Mockup para la gestión de grupos de pacientes]{Mockup para la gestión de grupos de pacientes}
        \label{fig:c4:mockup9}
    \end{minipage}
    \hfill\vline\hfill
    \begin{minipage}{.45\textwidth}
        \centering
        \includegraphics[width=0.8\linewidth, height=7cm]{Imagenes/anxA5-1.png}
        \caption[Mockup para la gestión de pautas (versión I)]{Mockup para la gestión de pautas (versión I)}
        \label{fig:c4:mockup10}
    \end{minipage}
\end{figure}

\paragraph{}
En la figura~\ref{fig:c4:mockup16} aparece un elemento genérico para incluir en la cabecera de la pantalla del usuario con información básica sobre este. Al pulsar sobre el botón modificar, se redireccionaría la a pantalla de la figura~\ref{fig:c4:mockup12}.

\begin{figure}[h]
    \centering
    \begin{minipage}{.45\textwidth}
        \centering
        \includegraphics[width=0.8\linewidth, height=7cm]{Imagenes/anxA5-2.png}
        \caption[Mockup para la gestión de pautas (versión II)]{Mockup para la gestión de pautas (versión II)}
        \label{fig:c4:mockup11}
    \end{minipage}
    \hfill\vline\hfill
    \begin{minipage}{.45\textwidth}
        \centering
        \includegraphics[width=0.8\linewidth, height=7cm]{Imagenes/anxA6.jpg}
        \caption[Mockup para la gestión de un paciente]{Mockup para la gestión de un paciente}
        \label{fig:c4:mockup12}
    \end{minipage}
\end{figure}

\paragraph{}
En la figura~\ref{fig:c4:mockup17} aparecen los dos elementos que se utilizarían tanto en la pantalla principal con el resumen de alertas de todos los pacientes como en la pantalla particular de cada paciente. El botón \textit{ver detalles} llevaría a una tabla o gráfica con el resumen de episodios citados.

\paragraph{}
En la figura~\ref{fig:c4:mockup18} aparece el resumen de los episodios de todos los pacientes en el periodo especificado en el segundo elemento de la figura anterior. Así, el elemento más a la izquierda sería un filtro de pacientes similar al que se encuentra en páginas web de compra de artículos en la que se podría filtrar a los mismos en función de la información que se tiene de ellos, así como se podría hacer usando el buscador horizontal que se encuentra en la parte superior de la imagen. Al pulsar sobre cualquiera de estas filas, se abriría la página con el resumen de episodios del paciente seleccionado. La diferencia entre ambas tablas reside principalmente en la utilización del espacio y en que la información provista sea menos o más visual: la opción más a la izquierda, a diferencia de lo que ocurre con la opción de la derecha, no incluiría fotografía y estaría pensada para ocupar menos espacio vertical por registro, lo que permitiría poder representar más registros en el mismo espacio respecto a la segunda opción.

\begin{figure}
    \centering
    \begin{minipage}{.45\textwidth}
        \centering
        \includegraphics[width=0.8\linewidth, height=7cm]{Imagenes/anxA7-1.png}
        \caption[Mockup para mostrar los episodios de un paciente (versión I)]{Mockup para mostrar los episodios de un paciente (versión I)}
        \label{fig:c4:mockup13}
    \end{minipage}
    \hfill\vline\hfill
    \begin{minipage}{.45\textwidth}
        \centering
        \includegraphics[width=0.8\linewidth, height=7cm]{Imagenes/anxA7-2.png}
        \caption[Mockup para mostrar los episodios de un paciente (versión II)]{Mockup para mostrar los episodios de un paciente (versión II)}
        \label{fig:c4:mockup14}
    \end{minipage}
\end{figure}

\paragraph{}
En la figura~\ref{fig:c4:mockup17} aparece el resumen de los episodios para todos los pacientes. Esto se incluiría en la página principal. La información es la misma, lo único que varía es su representación.

\begin{figure}
    \centering
    \begin{minipage}{.45\textwidth}
        \centering
        \includegraphics[width=0.8\linewidth, height=7cm]{Imagenes/anxA7-3.png}
        \caption[Mockup para mostrar los episodios de un paciente (versión III)]{Mockup para mostrar los episodios de un paciente (versión III)}
        \label{fig:c4:mockup15}
    \end{minipage}
\end{figure}


\begin{figure}[!htbp]
    \centering
    \includegraphics[scale=0.3]{Imagenes/anxA8.jpg}
    \caption[Mockup para mostrar la información básica del paciente]{Mockup para mostrar la información básica del paciente}
    \label{fig:c4:mockup16}
\end{figure}

\begin{figure}[!htbp]
    \centering
    %width=3cm, height=3cm
\begin{frame}{Frame Title}
    
\end{frame}[]    \includegraphics[scale=0.3]{Imagenes/anxA9.jpg}
    \caption[Mockup mostrar alertas de episodios de ira]{Mockup mostrar alertas de episodios de ira}
    \label{fig:c4:mockup17}
\end{figure}

\begin{figure}[!htbp]
    \centering
    %width=3cm, height=3cm
    \includegraphics[scale=0.3]{Imagenes/anxA10.jpg}
    \caption[Mockup para mostrar los filtros y las dos opciones para mostrar la tabla de pacientes]{Mockup para mostrar los filtros y las dos opciones para mostrar la tabla de pacientes}
    \label{fig:c4:mockup18}
\end{figure}

\begin{figure}
    \centering
    \begin{minipage}{.45\textwidth}
        \centering
        \includegraphics[width=0.8\linewidth, height=7cm]{Imagenes/anxA11-1.png}
        \caption[Mockup para mostrar el resumen de alertas de los pacientes (versión I)]{Mockup para mostrar el resumen de alertas de los pacientes (versión I)}
        \label{fig:c4:mockup19}
    \end{minipage}
    \hfill\vline\hfill
    \begin{minipage}{.45\textwidth}
        \centering
        \includegraphics[width=0.8\linewidth, height=7cm]{Imagenes/anxA11-2.png}
        \caption[Mockup para mostrar el resumen de alertas de los pacientes (versión II)]{Mockup para mostrar el resumen de alertas de los pacientes (versión II)}
        \label{fig:c4:mockup20}
    \end{minipage}
\end{figure}

\paragraph{}
Una vez creados los prototipos en papel de la aplicación web, nos reunimos con la terapeuta para mostrarselos y buscar el diseño que mejor se ajustase a las necesidades de los usuarios finales. Las principales conclusiones que se obtuvo de esta reunión fueron las siguientes:

\begin{itemize}
    \item El método de autentificación del terapeuta para iniciar sesión con la tupla correo electrónico y contraseña es apropiado.
    \item A la hora de registrar un paciente,es apropiado poder incluir las fotos de pacientes para mejorar la visualización. El correo electrónico no es necesario y el grupo al que pertenece cada paciente es prescindible. Para futuras versiones podría ser interesante agrupar a los pacientes y obtener resultados globales de estos grupos pero para esta versión se acaba decidiendo que es mejor centrarse en una solución más sencilla. Por ello, la figura~\ref{fig:c4:mockup3} debe suprimir la selección del grupo y la pantalla de la figura~\ref{fig:c4:mockup4} no tiene que aparecer.
    \item Entre el primer y segundo elemento de la figura~\ref{fig:c4:mockup18} para representar las pautas de cada paciente, se considera que la solución tiene que ser algo intermedio: conservar la estructura del primer elemento de dividir las pautas en diversas tablas según su grado de intensidad (con un termómetro en paralelo para poder correlacionar fácilmente la intensidad de cada pauta) pero añadiendo columnas adicionales a las pautas como por ejemplo el porcentaje de éxito, tal y como se ve en el segundo elemento.
    \item Se considera apropiado que en el resumen de los episodios de ira figure el tiempo que se mantuvo el paciente en cada uno de los estados, tal y como aparece en el primer elemento de la figura~\ref{fig:c4:mockup14}.
    \item La información de la pauta recomendada para cada episodio, tal y como aparece en el tercer elemento de la figura~\ref{fig:c4:mockup14} se considera apropiada, pero es incompleto, ya que para un mismo estado al paciente se le puede haber recomendado más de una pauta. El paciente puede haber descartado pautas previas, y esta información es importante que quede registrada.
    \item La figura~\ref{fig:c4:mockup15} muestra la representación los episodios en formato histograma, lo que es correcto. El gráfico de tarta que aparece en la gráfica ~\ref{fig:c4:mockup20} no aporta información útil, porque cómo se ha desarrollado el episodio es relevante, como por ejemplo el tiempo que ha pasado en cada estado. El segundo elemento de la figura~\ref{fig:c4:mockup7} hace una aproximación a esto pero es incompleto, puesto que es necesario que sea una gráfica interactiva que permita poder ampliarla para por ejemplo poder seleccionar solo un episodio e ir viéndolos uno a uno. Para este fin, la terapeuta propone basarse en el ejemplo de las gráficas de los desfibriladores.
    \item Las figuras~\ref{fig:c4:mockup17} y ~\ref{fig:c4:mockup18} son elementos adicionales que sirven para decorar las pantallas que se iban generando. La figura~\ref{fig:c4:mockup17} se considera apropiada mientras que el primer elemento de la figura~\ref{fig:c4:mockup18} se considera que no aporta información útil para la terapeuta.
    \item En la figura~\ref{fig:c4:mockup19} aparecen dos modelos de tablas y una serie de filtros para poder buscar los pacientes en dichas tablas. Los filtros se consideran apropiados, y ambos formatos de tablas también a excepción de las columnas que indican el número de alertas de cada una de las intensidades de la ira de cada paciente, ya que esta información sin estar agrupada en episodios y desglosada en fechas, no aporta información relevante al terapeuta.
\end{itemize}

\subsection{Segunda iteracción}

\paragraph{}
Para esta segunda iteracción, se usaron prototipos de alta fidelidad con Android y HTML para que estos fueran lo más similares posibles al resultado final de la aplicación. Esto se debe a que, tras la primera iteracción, se pudieron extraer los suficientes requisitos como para poder avanzar en detalles más concretos de la aplicación final.

\subsubsection{Diseño de la aplicación móvil de los pacientes}

\paragraph{}
Las pantallas de esta segunda fase para el móvil son las siguientes:

\begin{itemize}
    \item En la figura~\ref{fig:c4:mockup21} aparece la pantalla para introducir el código para sincronizar la pulsera y el móvil con el paciente. Es la equivalente de la fase anterior que aparece en la figura~\ref{fig:c4:mockup1}.
    \item En la figura~\ref{fig:c4:mockup22} podemos ver la pantalla para sincronizar el dispositivo durante su primer uso.
\end{itemize}

\begin{figure}[h]
    \centering
    \begin{minipage}{.45\textwidth}
        \centering
        \includegraphics[width=0.8\linewidth, height=7cm]{Imagenes/anxA12.png}
        \caption[Mockup para sincronizar el dispositivo]{Mockup para sincronizar el dispositivo}
        \label{fig:c4:mockup21}
    \end{minipage}
    \hfill\vline\hfill
    \begin{minipage}{.45\textwidth}
        \centering
        \includegraphics[width=0.8\linewidth, height=7cm]{Imagenes/anxA13.png}
        \caption[Mockup para calibrar el dispositivo]{Mockup para calibrar el dispositivo}
        \label{fig:c4:mockup22}
    \end{minipage}
\end{figure}

\begin{itemize}
    \item En la figura~\ref{fig:c4:mockup21} encontramos la que será la pantalla principal, donde aparece la pauta recomendada, el nivel de la ira representado mediante un termómetro horizontal, la posibilidad de cambiar de pauta y de enviar comentarios sobre la pauta en cuestión.
\end{itemize}

\begin{figure}[h]
    \centering
    \begin{minipage}{.45\textwidth}
        \centering
        \includegraphics[width=0.8\linewidth, height=7cm]{Imagenes/anxA14.png}
        \caption[Mockup de la pantalla principal]{Mockup de la pantalla principal}
        \label{fig:c4:mockup23}
    \end{minipage}
    \hfill\vline\hfill
    \begin{minipage}{.45\textwidth}
        \centering
        \includegraphics[width=0.8\linewidth, height=7cm]{Imagenes/anxA15.png}
        \caption[Mockup del histórico de episodios de un paciente]{Mockup del histórico de episodios de un paciente}
        \label{fig:c4:mockup24}
    \end{minipage}
\end{figure}

\begin{itemize}

    \item En la figura~\ref{fig:c4:mockup24} se encuentra una pantalla de resumen de la evolución seguida por el paciente. Esto es una innovación respecto a la anterior versión puesto que la terapeuta comentó que esta información puede servir para motivar al paciente al poder verificar en un histograma cómo se han reducido sus episodios de ira.

\begin{figure}[h!]
    \centering
    \begin{minipage}{.45\textwidth}
        \centering
        \includegraphics[width=0.8\linewidth, height=7cm]{Imagenes/anxA16.png}
        \caption[Mockup del menú de la aplicación]{Mockup del menú de la aplicación}
        \label{fig:c4:mockup25}
    \end{minipage}
\end{figure}

    \item En la figura~\ref{fig:c4:mockup25} se puede ver la manera que se utilizaría para cambiar entre las pantallas. En lugar de tener un menú horizontal que quite espacio para el resto de pantallas, este menú se desplegará como se hace en aplicaciones como Telegram o Whatsapp mediante el pulsado en un bocadillo a la izquierda de la barra horizontal superior o mediante el desplazamiento de la pantalla de izquierda a derecha.
\end{itemize}

\paragraph{}
Al revisar este diseño con la terapeuta, se obtuvieron las siguientes conclusiones:

\begin{itemize}
    \item Las pantallas que se muestran se ajustan bastante a lo esperado por parte de la terapeuta, pero en ellas aparece el concepto de alertas que debe desaparecer. Un episodio es un conjunto cronológicamente ordenado de alertas que empieza en el estado de reposo y acaba en el estado de reposo. Esa debe ser la unidad mínima con la que el terapeuta debe trabajar.
    \item El termómetro de la imágen debe ir al revés: el elemento de mayor grosor tiene que ir a la izquierda, que representará el menor nivel de ira.
    \item Se debe incluir un widget en la aplicación en el que aparezca un pequeño termómetro con el nivel de ira del paciente en ese momento para que el paciente pueda ver su estado actual de la ira sin necesidad de abrir la aplicación.
    \item Se deben incluir mensajes de alertas motivacionales tipo \textit{toast} como refuerzo positivo para el paciente cuando se detecte una disminución continuada de sus episodios de ira.
    \item A la hora de interpretar la actividad fisiológica del paciente, es necesario discernir entre actividades que puedan generar mayor activación que no impliquen episodios de ira (como al realizar ejercicio físico o al mantener relaciones sexuales) de aquellas que sean fruto de un estado de ira.
    \item En esta interacción se ha echado en falta poder ver cómo quedaría la pantalla principal de la figura~\ref{fig:c4:mockup23} cuando el paciente no está experimentando un episodio de ira.
    \item En la pantalla principal que aparece en la figura~\ref{fig:c4:mockup23} es necesario que no solo aparezca la pauta que se recomienda seguir para reducir el nivel de ira, sino que también es necesario que el paciente pueda seleccionar entre una lista de opciones la causa principal que ha generado la ira.
    \item En la pantalla que aparece en la figura~\ref{fig:c4:mockup23} hay que añadir una función para que, después de un tiempo, pregunte al usuario si ha seguido la pauta que se ha recomendado. Si responde afirmativamente y el paciente no se encuentra en estado de reposo, se le mostrarán más pautas acordes a su nivel actual de ira. Si responde negativamente, se seguirá el mismo procedimiento pero preguntando al paciente por qué no ha seguido la pauta antes de mostrar la siguiente pauta. Esta información será de utilidad en la consulta para poder afinar el conjunto de pautas asociado a cada paciente.
\end{itemize}

\subsubsection{Diseño de la aplicación web del terapeuta}
\paragraph{}
En este caso, se diseñaron exclusivamente las pantallas que pudieran ser fruto de mayor controversia, eliminando pantallas estándar como las de inicio de sesión o modificación de pacientes, puesto que la interfaz de estas funcionalidades ya quedó cerrada en la iteracción anterior.

\paragraph{}
En la figura ~\ref{fig:c4:mockup26} podemos ver la pantalla principal, que sería la implementación de los elementos de la figura ~\ref{fig:c4:mockup11}.

\begin{figure}[h]
    \centering
    %width=3cm, height=3cm
    \includegraphics[height=7.5cm, width=\textwidth]{Imagenes/anxA17.png}
    \caption[Mockup de la pantalla principal de la web]{Mockup de la pantalla principal de la web}
    \label{fig:c4:mockup26}
\end{figure}

\paragraph{}
En las figuras~\ref{fig:c4:mockup27} y~\ref{fig:c4:mockup28} se puede ver la pantalla en la que aparecería el resumen de cada paciente mediante una gráfica y una tabla de formato acordeón a continuación con información detallada de cada uno de los estados del episodio por los que ha pasado el paciente.

\begin{figure}[!htbp]
    \centering
    \includegraphics[height=7.5cm, width=\textwidth]{Imagenes/anxA18.png}
    \caption[Mockup de la pantalla para ver los episodios (I/II)]{Mockup de la pantalla para ver los episodios (I/II)}
    \label{fig:c4:mockup27}
\end{figure}

\begin{figure}[!htbp]
    \centering
    \includegraphics[height=7.5cm, width=\textwidth]{Imagenes/anxA19.png}
    \caption[Mockup de la pantalla para ver los episodios (II/II)]{Mockup de la pantalla para ver los episodios (II/II)}
    \label{fig:c4:mockup28}
\end{figure}

\paragraph{}
En la figura ~\ref{fig:c4:mockup25} se puede ver la implementación de la pantalla ~\ref{fig:c4:mockup5} en la versión en la que las pautas están separadas por el nivel de activación.

\begin{figure}[!htbp]
    \centering
    %width=3cm, height=3cm
    \includegraphics[height=7.5cm, width=\textwidth]{Imagenes/anxA20.png}
    \caption[Mockup de la pantalla para ver las pautas]{Mockup de la pantalla para ver las pautas}
    \label{fig:c4:mockup31}
\end{figure}

\paragraph{}
Tras la reunión con la terapeuta, los puntos que se sacaron en claro fueron los siguientes:

\begin{itemize}
    \item Las pantallas de las figura~\ref{fig:c4:mockup17} y~\ref{fig:c4:mockup18}  siguen teniendo en las tablas y en la gráfica alertas en lugar de eventos, por lo que los elementos que aparezcan en la web deben ser el conjunto de eventos del paciente en el que, al seleccionar algún episodio concreto, se pueda ver la información detallada de dicho episodio.
    \item La tabla de la figura~\ref{fig:c4:mockup18} debe ser interactiva. Se insiste en emular el diseño de los desfibriladores. La tabla de la figura~\ref{fig:c4:mockup19} que aparece debajo indicando las pautas seguidas en los distintos momentos es correcta, pero en el diseño final tiene que ser más detallada.
    \item La pantalla de la figura~\ref{fig:c4:mockup31} es una representación adecuada de las pautas almacenadas.
    \item Se echa en falta definir la pantalla en la que se añaden las pautas de cada paciente. En estas es necesario crear un sistema para poder heredar pautas de manera grupal para, a partir del paquete de pautas que se ha heredado, poder personalizarlas para cada paciente.
\end{itemize}

%TODO: esto probablemente acabe en otra sección mucho más detallada en la que se expliquen todos los aspectos técnicos de la solución final.
\section{Implementación}
\label{sec:c4:impl}
\paragraph{}
A partir de las conclusiones extraidas en las dos anteriores iteracciones, se implementó una solución que cumpliese con todos los requisitos extraídos de las reuniones con la persona experta. A continuación se encuentra el resultado final de este proceso.

\subsection{Diseño de la aplicación móvil de los pacientes}
[TODO]

\subsection{Diseño de la aplicación web del terapeuta}
\paragraph{}
En la aplicación web, se han distinguido cuatro elementos principales: las pautas, los grupos de pautas (que facilitan la asignación de pautas a los pacientes), los pacientes y los episodios. En el diseño de la web, se ha seguido el criterio de que, partiendo de estos cuatro elementos, haya cuatro tipos de pantallas: registro, visualización, modificación y búsqueda del elemento. El único de los cuatro elementos que no consta de estos cuatro tipos de pantallas es el de los episodios. Los episodios están vinculados a cada paciente y son inmutables. Esta información proviene de la aplicación móvil y sólo se podrá visualizar y filtrar en búsquedas de episodios. En ningún caso se podrá registrar o modificar episodios de pacientes desde la propia web, ya que esta información debe provenir de la transformación de las sensorizaciones realizadas en la pulsera. A continuación, se pueden ver todas las pantallas que se han incluido en la página web del terapeuta.

\paragraph{}
En la figura~\ref{fig:c4:impl1} se puede ver la pantalla para registrar un terapeuta. El único campo unívoco es el del correo electrónico.

\paragraph{}
En la figura~\ref{fig:c4:impl2} se muestra la pantalla para iniciar sesión. Es a la página a la que se redireccionan todas las peticiones que se realicen a páginas válidas del dominio que requieran que el usuario esté identificado pero no lo esté. En caso de marcar la casilla "recordar usuario", se guardará una cookie en el navegador con la sesión del mismo.

\begin{figure}[h]
    \centering
    %width=3cm, height=3cm
    \includegraphics[height=7cm, width=\textwidth]{Imagenes/15-registrarTerapeuta.png}
    \caption[Registrar terapeuta en la versión final de la web]{Registrar terapeuta en la versión final de la web}
    \label{fig:c4:impl1}
\end{figure}

\begin{figure}[h]
    \centering
    \includegraphics[height=7cm, width=\textwidth]{Imagenes/14-iniciarSesion.png}
    \caption[Iniciar sesión en la versión final de la web]{Iniciar sesión en la versión final de la web}
    \label{fig:c4:impl2}
\end{figure}

\paragraph{}
[TODO: página principal de la web]

\paragraph{}
En las figuras~\ref{fig:c4:impl4} y ~\ref{fig:c4:impl5} se puede ver cómo se registraría un paciente. Una vez se pulsa sobre el botón de registrar paciente en la figura~\ref{fig:c4:impl4}, se comprobarían si no se está incluyendo un paciente duplicado en la base de datos y, si no es así, se pasaría a la pantalla de la figura~\ref{fig:c4:impl5}. Esta es una pantalla para vincular el paciente que se ha registrado con el dispositivo móvil del paciente. Para ello, el paciente deberá introducir en la aplicación de Android el código que aparecerá por pantalla al terapeuta. Este código se eliminará de la base de datos (y, por tanto, se invalidará) tras cinco minutos para evitar que un atacante pueda hacer ataques de fuerza bruta y acabe consiguiendo emparejarse con algún paciente, suplantando su identidad. Este método de emparejamiento está diseñado dando por sentado que el paciente acudirá a la consulta con su dispositivo móvil y que el terapeuta procederá a registrarle en la aplicación en la propia consulta.

\begin{figure}[H]
    \centering
    \includegraphics[height=7cm, width=\textwidth]{Imagenes/14-iniciarSesion.png}
    \caption[Página principal en la versión final de la web (TODO)]{Página principal en la versión final de la web (TODO)}
    \label{fig:c4:impl3}
\end{figure}

\begin{figure}[H]
    \centering
    \includegraphics[height=7cm, width=\textwidth]{Imagenes/1-registrarPaciente.png}
    \caption[Página de registro de pacientes en la versión final de la web]{Página de de registro de pacientes en la versión final de la web}
    \label{fig:c4:impl4}
\end{figure}

\begin{figure}[H]
    \centering
    \includegraphics[height=7cm, width=\textwidth]{Imagenes/1-registrarPaciente-2.png}
    \caption[Página de sincronización del dispositivo en el registro de pacientes en la versión final de la web]{Página de sincronización del dispositivo en el registro de pacientes en la versión final de la web}
    \label{fig:c4:impl5}
\end{figure}

\paragraph{}
En la figura~\ref{fig:c4:impl6} se puede ver el buscador de pacientes. Si no se rellena ningún campo, se mostrarán todos los pacientes registrados del terapeuta. Si se rellenan los campos, se realizará una consulta con la suma de condiciones de los campos seleccionados. La tabla de pacientes incluye un paginador, igual que en el resto de tablas de la web, que se hace visible cuando la consulta devuelve más registros que el máximo número de páginas por tabla establecido (es decir, quince registros).

\paragraph{}
En la figura~\ref{fig:c4:impl7} se puede ver un paciente concreto de la web. Inicialmente, esta página y la página de la figura~\ref{fig:c4:impl8} eran una sola página en la que, mediante un botón con dos posiciones, se podía elegir si los campos eran editables (como en la figura~\ref{fig:c4:impl8}) o no (como en la figura~\ref{fig:c4:impl7}).

\paragraph{}
En la figura~\ref{fig:c4:impl8} aparece la pantalla para editar la información de un paciente, que viene precargada con los valores de dicho paciente. Inicialmente se incluía junto con los desplegables de pautas y grupos una tabla con todas las pautas que se fueran seleccionando así como las que se fueran creando específicamente para el paciente en esta misma pantalla. Esta tabla incluía una columna extra para poder desvincular cada pauta de cada fila de la tabla, de la misma manera que se podía y se pueden desvincular las pautas del seleccionable. Como se consideraba que esta opción podía ser confusa para el usuario al incluir demasiada información en una sola pantalla, se decidieron dejar exclusivamente los seleccionables, aunque eso implicase no poder ver en la pantalla de edición de los pacientes las características concretas de las pautas seleccionadas (en la tabla aparecían los niveles de alerta a los que estaban vinculados, mientras que en el seleccionable, esa información no aparece).

\begin{figure}[H]
    \centering
    \includegraphics[height=7cm, width=\textwidth]{Imagenes/2-verPacientes.png}
    \caption[Página para ver los pacientes en la versión final de la web]{Página para ver los pacientes en la versión final de la web}
    \label{fig:c4:impl6}
\end{figure}

\begin{figure}[H]
    \centering
    \includegraphics[height=7cm, width=\textwidth]{Imagenes/3-verPaciente.png}
    \caption[Página para ver un paciente concreto en la versión final de la web]{Página para ver un paciente concreto en la versión final de la web}
    \label{fig:c4:impl7}
\end{figure}

\begin{figure}[H]
    \centering
    \includegraphics[height=7cm, width=\textwidth]{Imagenes/4-editarPaciente.png}
    \caption[Página para editar un paciente concreto en la versión final de la web]{Página para editar un paciente concreto en la versión final de la web}
    \label{fig:c4:impl8}
\end{figure}

\paragraph{}
A la página de la figura~\ref{fig:c4:impl9} se accede a partir de la pantalla de la figura~\ref{fig:c4:impl7} o de la figura~\ref{fig:c4:impl6}, que incluyen enlaces para ver los episodios de los correspondientes pacientes. Esta pantalla varía exclusivamente respecto a la pantalla de la figura~\ref{fig:c4:impl10} en la parte superior de la página, que incluye la información para identificar al paciente del que se están consultando los episodios de ira mientras que en la figura~\ref{fig:c4:impl10}, al ser una pantalla genérica, en su lugar se incluye un seleccionable de los pacientes del terapeuta. A esta última pantalla se accede exclusivamente a través del elemento del menú principal \textit{Ver episodios}.

\begin{figure}[H]
    \centering
    \includegraphics[height=7cm, width=\textwidth]{Imagenes/5-verEpisodios.png}
    \caption[Página para ver los episodios de un paciente fijo en la versión final de la web]{Página para ver los episodios de un paciente fijo versión en la final de la web}
    \label{fig:c4:impl9}
\end{figure}

\begin{figure}[H]
    \centering
    \includegraphics[height=7cm, width=\textwidth]{Imagenes/6-verEpisodiosGenerico.png}
    \caption[Página para ver los episodios de un paciente variable versión en la final de la web]{Página para ver los episodios de un paciente variable en la versión final de la web}
    \label{fig:c4:impl10}
\end{figure}

\paragraph{}
Cuando se selecciona un episodio concreto en las filas de la tabla de las pantallas de las figuras~\ref{fig:c4:impl9} o~\ref{fig:c4:impl10}, se envía por parámetros a la pantalla de la figura~\ref{fig:c4:impl11} el paciente seleccionado y las fechas y horas de inicio y de fin del episodio. Con esta información, se recoge toda la información de ese episodio y se muestra en una gráfica y en una tabla que adicionalmente contiene las pautas sugeridas en cada momento y los comentarios opcionales rellenos por el usuario. Inicalmente, la gráfica que aparece en la figura~\ref{fig:c4:impl11} se incluyó en las pantallas para ver todos los episodios de los pacientes, pero se consideró que, como en el medio plazo un paciente seguramente enviaría bastante información de sus episodios de ira, esta gráfica perdería su interés, mientras que al incluirla en la pantalla de la figura~\ref{fig:c4:impl11}, como contendría significativamente menos información, esta gráfica podría ser útil para ver de manera visual el tiempo que ha permanecido el paciente en cada uno de los estados de la ira.

\begin{figure}[h]
    \centering
    \includegraphics[height=7cm, width=\textwidth]{Imagenes/7-verUnEpisodio.png}
    \caption[Página para ver un episodio concreto de un paciente en la versión final de la web]{Página para ver un episodio concreto de un paciente versión en la final de la web}
    \label{fig:c4:impl11}
\end{figure}

%%TODO: registrar pauta
\paragraph{}
En la figura~\ref{fig:c4:impl12} se puede ver la pantalla para registrar una pauta. En esta misma pantalla se puede también asociar la nueva pauta a pacientes o englobarla dentro de un grupo de pautas.

\paragraph{}
En la figura~\ref{fig:c4:impl13} se muestra el buscador de pautas. El funcionamiento es idéntico al de la página de la figura~\ref{fig:c4:impl6}.

\begin{figure}[H]
    \centering
    \includegraphics[height=7cm, width=\textwidth]{Imagenes/8-registrarPauta.png}
    \caption[Página para registrar una pauta en la versión final de la web]{Página para registrar una pauta en la versión final de la web}
    \label{fig:c4:impl12}
\end{figure}

\begin{figure}[H]
    \centering
    \includegraphics[height=7cm, width=\textwidth]{Imagenes/9-verPautas.png}
    \caption[Página para ver las pautas en la versión final de la web]{Página para ver las pautas en la versión final de la web}
    \label{fig:c4:impl13}
\end{figure}

\paragraph{}
En la página de la figura~\ref{fig:c4:impl14} se puede ver la información de una pauta. Esta página incluye un enlace para la página de edición de la pauta.

\paragraph{}
En la página de la figura~\ref{fig:c4:impl15} se muestra el buscador de grupos de pactos, cuyo funcionamiento es idéntico a los buscadores de pacientes y pautas que aparecen las figuras~\ref{fig:c4:impl6} y~\ref{fig:c4:impl13}.

\begin{figure}[H]
    \centering
    \includegraphics[height=7cm, width=\textwidth]{Imagenes/10-verPauta.png}
    \caption[Página para ver una pauta en la versión final de la web]{Página para ver una pauta en la versión final de la web}
    \label{fig:c4:impl14}
\end{figure}

\begin{figure}[H]
    \centering
    \includegraphics[height=7cm, width=\textwidth]{Imagenes/11-verGrupos.png}
    \caption[Página para ver los grupos de pautas en la versión final de la web]{Página para ver los grupos de pautas en la versión final de la web}
    \label{fig:c4:impl15}
\end{figure}

\begin{figure}[H]
    \centering
    \includegraphics[height=7cm, width=\textwidth]{Imagenes/12-verGrupo.png}
    \caption[Página para ver un grupo de pautas en la versión final de la web]{Página para ver un grupo de pautas en la versión final de la web}
    \label{fig:c4:impl16}
\end{figure}

\begin{figure}[H]
    \centering
    \includegraphics[height=7cm, width=\textwidth]{Imagenes/13-editarGrupo.png}
    \caption[Página para editar un grupo de pautas en la versión final de la web]{Página para editar un grupo de pautas en la versión final de la web}
    \label{fig:c4:impl17}
\end{figure}

\paragraph{}
En la figura~\ref{fig:c4:impl13} aparece la página para ver la información de un grupo de pautas. Al pulsar sobre el botón \textit{editar paciente}, se redireccionará a la página de la figura~\ref{fig:c4:impl17}, donde se podrá editar la información de dicho paciente.

\subsection{Flujo de datos}
\paragraph{}
La pulsera (tercer elemento de la figura~\ref{fig:c4:dfd}) es un dispositivo IOT que obtendrá las sensorizaciones del paciente (primer y segundo elemento de la figura~\ref{fig:c4:dfd}) cada [TODO] segundos. Concretamente, obtendrá los datos de la actividad electrodérmica, y el ritmo cardíaco. Para ello, se ha elegido el modelo [TODO: modelo y descripción del modelo de la pulsera haciendo referencia a la comparación de pulseras del estado del arte].

\paragraph{}
Estas sensorizaciones se enviarán por \ac{BLE} al dispositivo móvil con sistema operativo Android del paciente (conexión entre el tercer y cuarto elemento de la figura~\ref{fig:c4:dfd}). Este protocolo de comunicación utiliza el sistema de encriptación \ac{AES} (\citep{ble-aes}), lo que dificulta que un atacante que se encuentre dentro del radio de emisión de la pulsera inteligente pueda acceder a los datos sensorizados.

\paragraph{}
La conexión será gestionada mediante una aplicación Android creada para este proyecto (cuareto elemento de la figura~\ref{fig:c4:dfd}). En esta aplicación, se mostrarán pautas para el nivel de alerta detectado en el paciente (quinto elemento de la figura~\ref{fig:c4:dfd}), o, en caso de que no haya alertas, mensajes de refuerzo positivo.

\paragraph{}
Al conjunto de alertas que se producen desde el estado de reposo, el o los estados de alerta y la vuelta al estado de reposo se denomina episodio. Los episodios tienen aparejados una serie de pautas que se le recomiendan al paciente durante el episodio, que podrán ser descartadas por el paciente si no se consideran útiles y añadir comentarios que puedan ser después útiles en la consuta. El usuario adicionalmente deberá indicar si ha seguido o no la pauta recomendada (conexiones entre el elemento uno y cuatro de la figura~\ref{fig:c4:dfd}). En tanto en cuanto el paciente no vuelva al estado de reposo, se le deben de seguir mostrando pautas (quinto elemento de la figura~\ref{fig:c4:dfd}).

\paragraph{}
Estos episodios, creados a partir de los conjunto de pautas, alertas e interacciones del usuario con la aplicación (sexto elemento de la figura~\ref{fig:c4:dfd}), serán almacenados en dos bases de datos: la base de datos local de SQLite (séptimo elemento de la figura~\ref{fig:c4:dfd}) y la base de datos del servidor de MongoDB (octavo elemento de la figura ~\ref{fig:c4:dfd}). Es importante persistir los episodios en la base de datos del móvil para que el paciente también pueda ver en la aplicación su registro histórico de episodios. Los datos se enviarán al servidor mediante el protocolo de comunicación AMQP.

\begin{figure}[H]
    \centering
    %width=3cm, height=3cm
    \includegraphics[height=9cm, width=\textwidth]{Imagenes/dfd-all.png}
    \caption[Diagrama de flujo de datos de la aplicación]{Diagrama de flujo de datos de la aplicación}
    \label{fig:c4:dfd}
\end{figure}

\paragraph{}
Si el móvil tiene permisos de superusuario y la base de datos no está encriptada, una aplicación maliciosa podría acceder a estos datos de SQLite. Para evitar esto, se encriptará esta base de datos (séptimo elemento de la figura~\ref{fig:c4:dfd}) con el sistema [TODO: ponerlo cuando esté hecho].

\paragraph{}
Los episodios almacenados en la base de datos del servidor (octavo elemento de la figura~\ref{fig:c4:dfd}) podrán ser luego visualizados en la web (noveno elemento de la figura~\ref{fig:c4:dfd}) por parte del terapeuta (decimoprimer elemento de la figura~\ref{fig:c4:dfd}). El terapeuta podrá añadir nuevos pacientes, modificar las pautas de los pacientes a través de la web, información que luego se persistirá a su vez en la base de datos de los pacientes correspondientes (séptimo elemento de la figura~\ref{fig:c4:dfd}).

\paragraph{}
Para que un terapeuta pueda acceder a la web, antes tiene que registrarse e iniciar sesión. Los credenciales de los terapeutas se guardan en una base de datos de SQLite (décimo elemento de la figura~\ref{fig:c4:dfd}). Estos credenciales están separados del resto de registros de la web por motivos de ciberseguridad. Así, si se detecta un intento de acceso indebido a la base de datos con los credenciales de los terapeutas, el administrador de la web, podría a bloquear cautelarmente el acceso a la base de datos de MongoDB (octavo elemento de la figura~\ref{fig:c4:dfd}).